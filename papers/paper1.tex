%---------DO NOT EDIT THIS INDENTED SECTION
	% Preamble
	\documentclass[11pt,reqno,oneside,a4paper]{article}
	\usepackage[a4paper,includeheadfoot,left=35mm,right=35mm,top=00mm,bottom=30mm,headheight=40mm]{geometry} %sets up the margins
	%%%%%%%%%%%%%%%%%%%%%%%%%%%%%%%%%%%%%%%%%%%%%%%%%%%%%%%%%%%%%%%%%%%%%%%%%%%%%%%%
%
% This file contains some standard modifications to basic LaTeX2e and
% the article documentclass. DO NOT EDIT THIS FILE, but do look through
% and make use of the shorthands defined herein.
%
%%%%%%%%%%%%%%%%%%%%%%%%%%%%%%%%%%%%%%%%%%%%%%%%%%%%%%%%%%%%%%%%%%%%%%%%%%%%%%%%

% Standard packages
\usepackage{amssymb,amsmath,amsthm}
\usepackage{xcolor,graphicx}
\usepackage{verbatim}
\usepackage{hyperref}
% Layout of headers & footers
\usepackage{titling}
\usepackage{fancyhdr}
\pagestyle{fancy} \lhead{{\theauthor}} \chead{} \rhead{} \lfoot{} \cfoot{\thepage} \rfoot{}

% Hyphenation
\hyphenation{non-zero}

% Instroctor's email address
\newcommand{\InstEmail}{dave.smith@yale-nus.edu.sg}

% Theorem definitions in the amsthm standard
\newtheorem{thm}{Theorem}
\newtheorem{lem}[thm]{Lemma}
\newtheorem{sublem}[thm]{Sublemma}
\newtheorem{prop}[thm]{Proposition}
\newtheorem{cor}[thm]{Corollary}
\newtheorem{conc}[thm]{Conclusion}
\newtheorem{conj}[thm]{Conjecture}
\theoremstyle{definition}
\newtheorem{defn}[thm]{Definition}
\newtheorem{cond}[thm]{Condition}
\newtheorem{asm}[thm]{Assumption}
\newtheorem{ntn}[thm]{Notation}
\newtheorem{prob}[thm]{Problem}
\theoremstyle{remark}
\newtheorem{rmk}[thm]{Remark}
\newtheorem{eg}[thm]{Example}
\newtheorem*{hint}{Hint}

%% Mathmode shortcuts
% Number sets
\newcommand{\NN}{\mathbb N}              % The set of naturals
\newcommand{\NNzero}{\NN^0}              % The set of naturals including zero
\newcommand{\NNone}{\NN}                 % The set of naturals excluding zero
\newcommand{\ZZ}{\mathbb Z}              % The set of integers
\newcommand{\QQ}{\mathbb Q}              % The set of rationals
\newcommand{\RR}{\mathbb R}              % The set of reals
\newcommand{\CC}{\mathbb C}              % The set of complex numbers
\newcommand{\KK}{\mathbb C}              % An arbitrary field
% Modern typesetting for the real and imaginary parts of a complex number
\renewcommand{\Re}{\operatorname*{Re}} \renewcommand{\Im}{\operatorname*{Im}}
% Upright d for derivatives
\newcommand{\D}{\ensuremath{\,\mathrm{d}}}
% Make epsilons look more different from the element symbol
\renewcommand{\epsilon}{\varepsilon}
% Always use slanted forms of \leq, \geq
\renewcommand{\geq}{\geqslant} \renewcommand{\leq}{\leqslant}
% Shorthand for some relations
\newcommand{\po}{\preceq} \newcommand{\rel}{{\mathcal R}} \newcommand{\rels}{\mathbin{\scriptstyle{\mathcal R}}}
% Shorthand for "if and only if" symbol
\newcommand{\Iff}{\ensuremath{\Leftrightarrow}}
% Make bold symbols for vectors
\providecommand{\BVec}[1]{\mathbf{#1}}
% Barred forms of \oplus and \otimes to represent the descents of these binary operators
\newcommand{\oplusbar}{\mathbin{\ooalign{$\hidewidth\overline{\oplus}\hidewidth$\cr$\phantom{\oplus}$}}} \newcommand{\otimesbar}{\mathbin{\ooalign{$\hidewidth\overline{\otimes}\hidewidth$\cr$\phantom{\otimes}$}}}
% Mathematical operators used in Proof
\DeclareMathOperator{\sgn}{sgn}          % The signum of a real number
\DeclareMathOperator{\power}{\mathcal{P}} % The power set of a set
\DeclareMathOperator{\Id}{Id}            % The identity function
\DeclareMathOperator{\Fun}{Fun}          % The set of functions from one set to another
\DeclareMathOperator{\Perm}{Perm}        % The set of permutations on a set
\DeclareMathOperator{\GCD}{GCD}          % The greatest common divisor of two integers
\newcommand{\abs}[1]{\left\lvert#1\right\rvert} % The absolute value of a real number or modulus of a complex number, with automatically scaling delimiters
 % Use the standard texHead for this module. You should not edit this file.
	%%%%%%%%%%%%%%%%%%%%%%%%%%%%%%%%%%%%%%%%%%%%%%%%%%%%%%%%%%%%%%%%%%%%%%%%%%%%%%%%
%
% You can make any edits you like to this file. It is designed as a place
% for you to define your own macros that you want to use across many
% LaTeX documents. You can also define macros in a single document,
% but if you find yourself reusing the same macro in several documents,
% then it probably belongs in here. Look in the "%% Mathmode shortcuts"
% section of texHead-Proof-Standard.tex for some examples of how to define
% your own macros.
%
%%%%%%%%%%%%%%%%%%%%%%%%%%%%%%%%%%%%%%%%%%%%%%%%%%%%%%%%%%%%%%%%%%%%%%%%%%%%%%%%

% My own macros
\newcommand{\ominusbar}{\mathbin{\ooalign{$\hidewidth\overline{\ominus}\hidewidth$\cr$\phantom{\ominus}$}}}

 % Use your personal additional macros. You may edit this file.
	%---The following code defines the title, author, and date of the document.
	\title{Paper 1}
	\date{\today}   % Using \today automatically updates to the document's build date
%----------------------------------
%---------IF YOU WANT TO DEFINE YOUR OWN MACROS, YOU CAN DO SO EITHER IN ../texHead-Proof-Personal.tex OR FROM HERE ...

%---------... TO HERE

\author{Therion} %%%%%%%% EDIT THIS LINE. This should be your pseudonym from practice proof 3. You can find it in the feedback for that assignment on Canvas. If you can't find it, then please email the instructor to ask.

\begin{document}
\maketitle
\thispagestyle{fancy}

%-----------EDIT FROM HERE

\section{Mean-spirited} \label{sec:mean-spirited}

\begin{thm} \label{thm:mean-spirited}
	For all $a,b,c>0$,
	$$
		\frac{a^2}{bc} + \frac{b^2}{ca} + \frac{c^2}{ab} \geq 3.
	$$
\end{thm}

\begin{proof}[Invalid proof of theorem~\ref{thm:mean-spirited}]
	Suppose (for a contradiction) that
	$$
		\frac{a^2}{bc} + \frac{b^2}{ca} + \frac{c^2}{ab} < 3.
	$$
	Then this inequality must be true for the particular choice of $a=b=c=1>0$.
	But
	$$
		\frac{1^2}{1\times1} + \frac{1^2}{1\times1} + \frac{1^2}{1\times1} = 1+1+1 = 3 \not< 3.
	$$
	This contradicts our assumption, so the theorem is true.
\end{proof}

\noindent\emph{Discussion}:
%%%%%%%%%%%%%%%%%%%%%%%%%%%%%%%%%
%
% THE ABOVE CLAIMED PROOF IS INCORRECT.
% WRITE A FEW SENTENCES TO EXPLAIN THE ERROR.
%
% 1. Is the theorem true or false. If the theorem is false, can you disprove it?
% 2. Whether or not the theorem is true, is the proof correct or incorrect? If there are any errors in the proof, point them out and, if possible, correct.
% 3. Are there any parts of the proof that could be made more efficient?
%
%%%%%%%%%%%%%%%%%%%%%%%%%%%%%%%%%


\noindent The theorem is true. However, the proof provided for the theorem is invalid. To correct the invalid proof, we note that the contrary statement is written incorrectly since it did not negate the universal quantifier. 
Consequently, the incorrect proof only provided a particular counter-example to form a contradiction, i.e. that there \emph{exists} a particular $a, b, c > 0$ such that 

$$\frac{a^2}{bc} + \frac{b^2}{ca} + \frac{c^2}{ab} < 3$$

\noindent However, the correct contrary statement should be: ``Suppose (for a contradiction) that $\exists  a, b, c > 0$ such that

$$\frac{a^2}{bc} + \frac{b^2}{ca} + \frac{c^2}{ab} < 3."$$

\noindent Thus, the invalid proof should have provided a counter-example for \emph{all} $a,b,c>0$ to form the contradiction to the existential quantifier.
To establish a valid contradiction to the correct contrary statement, we divide both sides of the inequality by $3$ to get 

$$\frac{\frac{a^2}{bc} + \frac{b^2}{ca} + \frac{c^2}{ab}}{3} < 1. $$

\noindent The LHS of the above inequality is the arithmetic mean (A.M.) of $\frac{a^2}{bc} + \frac{b^2}{ca} + \frac{c^2}{ab}$. We note that the geometric mean (G.M.) of the numbers is 
$\sqrt[3]{\frac{a^2}{bc} + \frac{b^2}{ca} + \frac{c^2}{ab}} = 1$. Considering the A.M.-G.M. inequality, A.M. $\geq$ G.M. Therefore, LHS should be $\geq 1$. This is a contradiction, and thus our assumption must be false. Hence,
for \emph{all} $a,b,c$, 

$$\frac{a^2}{bc} + \frac{b^2}{ca} + \frac{c^2}{ab} \geq 3.$$

\noindent This corrects the proof. The proof could be made more efficient by using a direct proof with the A.M.-G.M. inequality instead of proof by contradiction, as I will demonstrate in the correct proof.



\begin{proof}[Correct proof of theorem~\ref{thm:mean-spirited}]
	%%%%%%%%%%%%%%%%%%%%%%%%%%%%%%%%%
	%
	% HINTS
	%
	% Use the AM-GM inequality.
	%
	%%%%%%%%%%%%%%%%%%%%%%%%%%%%%%%%%	
We can prove theorem \ref{thm:mean-spirited} directly. Since we know that $\frac{a^2}{bc}$, $\frac{b^2}{ca}$, and $\frac{c^2}{ab}$ are positive terms, applying the A.M.-G.M. inequality to them yields:

$$\frac{\frac{a^2}{bc} + \frac{b^2}{ca} + \frac{c^2}{ab}}{3} \geq \sqrt[3]{\frac{a^2}{bc} + \frac{b^2}{ca} + \frac{c^2}{ab}}.$$

\noindent The RHS of the above equation simplifies to $1$, which implies

$$ \frac{\frac{a^2}{bc} + \frac{b^2}{ca} + \frac{c^2}{ab}}{3} \geq 1.$$

\noindent Multiplying both sides by 3, we have 

$$ \frac{a^2}{bc} + \frac{b^2}{ca} + \frac{c^2}{ab}} \geq 3 $$

\noindent which is theorem \ref{thm:mean-spirited}.
\end{proof}

\section{Rationals can always be expressed as ratios as arbitrary powers{\dots}or can they?}

\noindent\emph{The below claimed theorem and proof may or may not be correct.}

\begin{thm} \label{thm:power-rational}
	For each $m,n\in\NN$, and for every positive $x\in\QQ$, there exist $a,b\in\NN$ such that $x = a^m / b^n$.
\end{thm}

\begin{proof}
	By definition, each positive rational can be expressed as $a^1/b^1$ for some choice of naturals $a,b$.
	Suppose that $x\in\QQ$ is such that $x=a^m/b$ for some $m\in\NN$.
	Then
	$$
		x = \frac{a^{m+1}}{ab} = \frac{(a)^{m+1}}{(ab)^1} = \frac{\alpha^{m+1}}{\beta},
	$$
	using $\alpha=a\in\NN$ and $\beta=ab\in\NN$.
	Hence, by induction on $m$, for every $m\in\NN$, and for every positive $x\in\QQ$, there exist $a,b\in\NN$ such that $x = a^m / b$.
	
	Now suppose that $x\in\QQ$ is such that $x=a/b^n$ for some $n\in\NN$.
	Then
	$$
		x = \frac{ab}{b^{n+1}} = \frac{(ab)^1}{(b)^{n+1}} = \frac{\alpha}{\beta^{n+1}},
	$$
	using $\alpha=ab\in\NN$ and $\beta=b\in\NN$.
	Hence, by induction on $n$, for every $n\in\NN$, and for every positive $x\in\QQ$, there exist $a,b\in\NN$ such that $x = a / b^n$.
	
	It follows that for every $m,n\in\NN$, and for every positive $x\in\QQ$, there exist $a,b\in\NN$ such that $x = a^m / b^n$.
\end{proof}

\noindent\emph{Discussion}:
%%%%%%%%%%%%%%%%%%%%%%%%%%%%%%%%%
%
% DISCUSS THE ABOVE CLAIMED THEOREM AND PROOF.
% YOU SHOULD WRITE TWO OR THREE PARAGRAPHS TO DISCUSS THE ISSUES BELOW.
%
% 1. Is the theorem true or false. If the theorem is false, can you disprove it?
% 2. Whether or not the theorem is true, is the proof correct or incorrect? If there are any errors in the proof, point them out and, if possible, correct.
% 3. Are there any parts of the proof that could be made more efficient?
%
%%%%%%%%%%%%%%%%%%%%%%%%%%%%%%%%%
The theorem is false. This can be shown through a counter-example where $m = n = x = 2$. Using this counter-example, consider that the theorem claims that for \emph{all} $m,n \in \mathbb{N}$ and for
\emph{all} positive $x \in \mathbb{Q}$, 
$$2 = \frac{a^2}{b^2}$$
for some $a,b \in \mathbb{N}$. However, this implies $$\sqrt{2} = \frac{a}{b}.$$ This is absurd: on the RHS $a,b \in \mathbb{N}$ and thus $\frac{a}{b} \in \mathbb{Q}$, whereas on the LHS $\sqrt{2}$ is irrational as
previously proven. Therefore, no $a,b \in \mathbb{N}$ will satisfy the equality of the counter-example $2 = \frac{a^2}{b^2}$, and thus theorem \ref{thm:power-rational} is false. \\

\noindent The proof is incorrect. The logical flow of the incorrect proof is as follows. Firstly, it does induction on $m$ for $n=1$. Next, it does induction on $n$ for $m=1$. These
two inductions, considered separately, are valid. However, the final step of the proof claims that for \emph{every} $m,n \in \mathbb{N}$, and for \emph{every} positive $x \in \mathbb{Q}$, there exist $a, b \in \mathbb{N}$ 
such that $x = \frac{a^m}{b^n}$. This final statement does not logically follow from the two inductions, since an induction on $m$ with $n=1$ and an induction on $n$ with $m=1$ done separately does not imply induction
for \emph{all} $m$ and $n$. Thus, if theorem 
\ref{thm:power-rational} is expressed as the statement $P(m,n)$ for all $m,n \in \mathbb{N}$, then the incorrect proof has only proven $P(m,1)$ and $P(1,n)$ and not $P(m,n)$.\\

\noindent Since the theorem is false, it is not possible to prove it. However, a way to make this proof more efficient is to note that proving $P(1,n)$ is unnecessary once $P(1,1)$ and $P(m,1)$ are proven. Thus, the proof for the base case for the second induction is unnecessary.




\section{An irrootional number}

\begin{defn} \label{defn:rootional}
	The \emph{rootionals} (or \emph{rootional numbers}) are the natural roots of rational numbers.
	They are the answers to the question ``Of what should I multiply $n$ copies to get $x$?'', when $0<x\in\QQ$, and $n\in\NN$.
	The \emph{set of rootionals} is denoted $\mathbb{P}$.
	A real number which is not rootional is called \emph{irrootional}.
\end{defn}

\begin{lem} \label{lem:powers-of-1+sqrt2}
	% THE SYMBOLS $\phi$ AND $\psi$ SHOULD NOT APPEAR ANYWHERE IN THE STATEMENT OF THE THEOREM OR THE PROOF WHEN YOU SUBMIT.
	% REPLACE $\phi$ and $\psi$ WITH EXPLICIT FORMULAE.
	The sequences $(a_n)_{n\in\NN}$, $(b_n)_{n\in\NN}$ given by
	\begin{align*}
		a_1 &= 1 & a_{n+1} &= a_n + 2b_n, & n &\in\NN, \\
		b_1 &= 1 & b_{n+1} &= a_n + b_n, & n &\in\NN,
	\end{align*}
	satisfy both
	$$
		\left(1+\sqrt{2}\right)^n = a_n + b_n \sqrt{2} \qquad\mbox{and}\qquad a_n^2-2b_n^2 = (-1)^n,
	$$
	for all $n\in\NN$.
\end{lem}

\begin{proof}
	%%%%%%%%%%%%%%%%%%%%%%%%%%%%%%%%%
	%
	% HINTS
	%
	% First you have to complete the statement of the lemma.
	% This means that you have to replace both $\phi(a_n,b_n)$ and $\psi(a_n,b_n)$ with explicit formulae.
	% There is no need to show how you worked out what should replace $\phi$ and $\psi$, just enter the replacements into the statement of the lemma and then give the proof as if you always had always known what $\phi$ and $\psi$ are.
	% THE SYMBOLS $\phi$ AND $\psi$ SHOULD NOT APPEAR ANYWHERE IN THE STATEMENT OF THE LEMMA OR THE PROOF WHEN YOU SUBMIT.
	%
	% The proof is by induction.
	%
	%%%%%%%%%%%%%%%%%%%%%%%%%%%%%%%%%
	The proof of Lemma \ref{lem:powers-of-1+sqrt2} will be done by induction on $n$. Let $C(n)$ be the statement:
	$$``(1+\sqrt{2})^n = a_n + b_{n}\sqrt{2} \:\:\textrm{and}\:\: a_n^2 - 2b_n^2 = (-1)^n $$
	are satisfied by the sequences $(a_n)_{n\in\mathbb{N}}, (b_n)_{n\in\mathbb{N}}$" for all $n \in \mathbb{N}$. 
	Let $L(n)$ be the statement $``(1+\sqrt{2})^n = a_n + b_{n}\sqrt{2} "$ and
	$R(n)$ be the statement $``a_n^2 - 2b_n^2 = (-1)^n"$.\\
	
	\noindent For the base case, when $n=1$, $a_1 = 1$ and $b_1 = 1$. The LHS of $L(1)$ is $(1+\sqrt{2})^1 &= 1 + 1\cdot\sqrt{2} &= a_1 + b_1\sqrt{2} =$ RHS. Similarly, for $R(1)$, LHS = $a_1^2 - 2b_1^2 = 1^2 - 2\cdot 1^2 = -1
	= (-1)^1 =$ RHS. Hence, since both $L(1)$ and $R(1)$ are true, the base case $C(1)$ is true.\\
	
	\noindent For the inductive step, assume $C(n), L(n), R(n)$ are true for some $n\in\mathbb{N}$. To show $C(n) \implies C(n+1)$, we first show $L(n) \implies L(n+1)$ and $R(n) \implies R(n+1)$:
	
	\begin{align*}
	LHS \:\:\textrm{of}\:\: L(n+1) &= (1+\sqrt{2})^{n+1}\\
	&= (1+\sqrt{2})^n(1+\sqrt{2}) \\
	&= (a_n +b_n\sqrt2)(1+\sqrt2), \:\textrm{from}\:\:L(n)\\
	&= a_n + \sqrt{2}a_n + b_n\sqrt{2} + 2b_n\\
	&= a_{n+1} + b_{n+1}\sqrt{2} \\
	&= RHS\\
	LHS \:\:\textrm{of}\:\: R(n+1) &= a_{n+1}^2 - 2b_{n+1}^2\\
	&= (a_n + 2b_n)^2 - 2(a_n +b_n)^2\\
	&= a_n^2 + 4a_{n}b_{n} + 4b_n^2 -2(a_n^2 + 2a_nb_n + b_n^2)\\
	&= a_n^2 + 4b_n^2 -2a_n^2 - 2b_n^2\\
	&= 2b_n^2 - a_n^2\\
	&= -1 \cdot(a_n^2 -2b_n^2)\\
	&= -1 \cdot (-1)^n, \:\textrm{from}\:\:R(n)\\
	&= (-1)^{n+1} \\
	&= RHS
	\end{align*}
	
	\noindent Therefore, $L(n) \implies L(n+1)$ and $R(n) \implies R(n+1)$, and thus $C(n) \implies C(n+1)$. Because we know $C(1)$ is true, and $C(n+1)$ is true if $C(n)$ is true, by induction 
	$C(n)$ is true for all $n \in \mathbb{N}$.
\end{proof}

\begin{thm} \label{thm:irrootional-eg}
	The number $1+\sqrt{2}$ is irrootional.
\end{thm}

\begin{proof}
	This will be a proof by contradiction. Assume that $1+\sqrt2$ is rootional. Then, by definition \ref{defn:rootional}, $(1+\sqrt2)^n$ is a rational number for some $n\in\mathbb{N}$.
	From Lemma \ref{lem:powers-of-1+sqrt2}, we also know that $(1+\sqrt2)^n = a_n + b_n\sqrt2$.
	We can express this as follows:	
	\begin{align*}
	(1+\sqrt2)^n &= x, \textrm{where}\:\:x\in\mathbb{Q}\\
	a_n + b_n\sqrt2 &= x\\
	\sqrt2 &= \frac{x-a_n}{b_n}
	\end{align*}
	
	\noindent We observe from Lemma \ref{lem:powers-of-1+sqrt2} that the functions $a_{n+1} = a_n + 2b_n$ and $b_{n+1} = a_n + b_n$ are monotone increasing. Since $a_1 = b_1 = 1$, $a_{k+1} > a_k$ and $b_{k+1} > b_k$, we 
	know that $a_n, b_n \in\mathbb{N}$. Thus, since $x\in\mathbb{Q}$ and $a_n,b_n \in\mathbb{N}$, we know $\frac{x-a_n}{b_n} \in\mathbb{Q}$. This can be shown by expressing $x$ as 
	a fraction of two integers, $q$ as the numerator and $p$ as the denominator:
	$$\frac{x-a_n}{b_n} = \frac{\frac{q}{p} - a_n}{b_n} = \frac{q - p(a_n)}{p(b_n)}$
	Since all the terms in $\frac{q - p(a_n)}{p(b_n)}$ are integers, the numerator and denominator are integers and therefore by definition we know that $\frac{x-a_n}{b_n} \in\mathbb{Q}$ for some $n\in\mathbb{N}$.
	However, from a previous proof, we know that $\sqrt2$ is 
	irrational. This is a contradiction, and we thus know that our assumption is wrong. Hence, $1+\sqrt2$ is irrootional.
\end{proof}

\section{$n$-ary representations of integers}

\begin{defn} \label{defn:n-ary-sequence}
	For any integer $n\geq2$, a sequence $(a_j)_{j\in\NN^0}$ is called \emph{$n$-ary} if
	$$
		\{a_j:j\in\NN\}\subset\{0,1,\ldots,n-1\}.
	$$
\end{defn}

\begin{thm} \label{thm:n-ary-rep-int}
	For each $x\in\NN^0$, and for each integer $n\geq2$, there exists an eventually zero $n$-ary sequence $(a_j)_{j\in\NN^0}$ such that
	$$
		x = \sum_{j=0}^\infty a_j n^j.
	$$
\end{thm}

\begin{proof}
	Let $P(x)$ be the statement: ``For each $x\in\mathbb{N}^0$, and for each integer $n\geq2$, there exists an eventually zero n-nary sequence $(a_j)_{j\in\mathbb{N}^0}$ such that 
	$$x = \sum_{j=0}^{\infty}{a_{j}n^j}."$$
	Note that, because the sequence is eventually zero, the series is a finite sum and converges.
	Base case: the case $P(0)$ when $x=0$ is trivially true for all $n\geq2$, with the empty sum given by the sequence in which every term is zero.\\
	
	\noindent Suppose that, for some integer $x \geq 0$, for every integer $k$ such that $0\leq k \leq x$, there is an eventually zero n-nary sequence $(a_{k,j})_{j\in\mathbb{N}^0}$, such that
	$$k = \sum_{j=0}^{\infty}{a_{k,j}n^j}$$
	is true for all $n\geq2$.\\
	
	\noindent If $x+1$ is divisible by $n$, then $0\leq \frac{x+1}{n}\leq x, \frac{x+1}{n}\in\mathbb{N}^0$, for all integer $n \geq 2$. By the inductive hypothesis, there exists a sequence $(a_{\frac{x+1}{n}},j)_{j\in\mathbb{N}^0}$,
	which is eventually zero, for which $$\frac{x+1}{n} = \sum_{j=0}^{\infty}{a_{\frac{x+1}{n},j}n^j}$$
	
	\noindent for each integer $n\geq2$. Hence
	$$x+1=n \sum_{j=0}^{\infty}{a_{\frac{x+1}{n},j}n^j} = \sum_{j=0}^{\infty}{a_{\frac{x+1}{n},j}n^{j+1}} = \sum_{j=1}^{\infty}{a_{\frac{x+1}{n},j-1}n^{j}}.
	
	\noindent By defining $a_{x+1,j} = a_{\frac{x+1}{n},j-1}$ and $a_{x+1,0} = 0$, we get
	$$x+1=\sum_{j=0}^{\infty}{a_{x+1,j}n^j}.$$
	
	\noindent This new sequence is eventually zero because $(a_{\frac{x+1}{n}},j)_{j\in\mathbb{N}^0}$ was eventually zero. 
	This new sequence is n-nary because the original sequence was n-nary and the only additional term is 0.\\
	
	\noindent If $x+1$ is not divisible by $n$, then let $R$ be the remainder of $\frac{x+1}{n}$. Since $R$ is the remainder, we know $1\leq R\leq n-1$. By the Quotient-Remainder theorem,
	we also know that $0\leq\frac{x+1-R}{n}\leq x$ and $n \mid (x+1-R)$ for all integer $n \geq 2$.
	By the inductive hypothesis, there exists a sequence $(a_{\frac{x+1-R}{n},j\right{)}_{j\in\mathbb{N}^0}$ which is eventually zero such that 
	$$\frac{x+1-R}{n} = \sum^n_{j=0}a_{\frac{x+1-R}{n},j}n^j$$
	
	\noindent for each integer $n\geq2$. Hence 
	$$x+1 = \sum^n_{j=0}a_{\frac{x+1-R}{n},j}n^{j+1} + R = \sum^n_{j=1}a_{\frac{x+1-R}{n},j-1}n^j + R$$ 
	
	\noindent By defining $a_{(x+1-R, j)} = a_{(\frac{x+1-R}{n},j-1)}$ and $a_{(x+1-R,0)} = R$, since $0\leq R \leq n-1$, the sequence is still n-nary and we get
	$$x + 1 = \sum^\infty_{j=0}{a_{x+1-R,j}n^j$$
	
	\noindent This new sequence is eventually zero because $(a_{\frac{x+1-R}{n},j\right{)}_{j\in\mathbb{N}^0}$ was eventually zero.
	\noindent This new sequence is n-nary because the original sequence was n-nary and the only additional term is R, where $1\leq R\leq n-1$.\\
	
	\noindent Because $P(0)$ is true for \emph{all} integer $n\geq2$, and $P(x+1)$ is true under all the assumptions of $P(k)$ for \emph{all} integer $n\geq2$, by complete induction we know that $P(x)$ is true for all $x\in\mathbb{N}^0$ and for 
	all integer $n\geq2$.
	
	
\end{proof}

\end{document}



