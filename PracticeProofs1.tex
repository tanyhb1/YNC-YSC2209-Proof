%---------DO NOT EDIT THE FOLLOWING INDENTED SECTION
	% Preamble
	\documentclass[11pt,reqno,oneside,a4paper]{article}
	\usepackage[a4paper,includeheadfoot,left=35mm,right=35mm,top=00mm,bottom=30mm,headheight=40mm]{geometry} %sets up the margins
	\usepackage{amssymb,amsmath,amsthm}
	\usepackage{xcolor,graphicx}
	\usepackage{hyperref}

	\usepackage{titling}
	\usepackage{fancyhdr}
	\pagestyle{fancy}
	\lhead{{\theauthor}}
	\chead{}
	\rhead{}
	\lfoot{}
	\cfoot{\thepage}
	\rfoot{}

	% \usepackage{tikz}  % Uncomment this line iff you are using the tikz package to add drawings
	%---The following code sets up the way theorems are typeset and labled.
	\newtheorem{thm}{Theorem}
	\newtheorem{lem}[thm]{Lemma}
	\newtheorem{cor}[thm]{Corollary}
	\newtheorem{prop}[thm]{Proposition}
	\theoremstyle{definition}
	\newtheorem{defn}[thm]{Definition}
	\newtheorem{eg}[thm]{Example}
	\theoremstyle{remark}
	\newtheorem{rmk}[thm]{Remark}
	%---The following code defines a few extra commands that will be useful in some exercises.
	\newcommand{\abs}[1]{\lvert#1\rvert}     % Absolute value symbol
	\newcommand{\ZZ}{\mathbb Z}              % The set of integers
	\newcommand{\QQ}{\mathbb Q}              % The set of rationals
	\newcommand{\RR}{\mathbb R}              % The set of reals
	\newcommand{\NN}{\mathbb N}              % The set of reals
	\newcommand{\CC}{\mathbb C}              % The set of complex numbers
	\newcommand{\power}{\mathcal{P}}         % The power set of a set
	\newcommand{\Id}{\mathrm{Id}}            % The identity function
	\providecommand{\BVec}[1]{\mathbf{#1}}   % Bold font for vectors
	\DeclareMathOperator{\gon}{gon}          % A polygon
	\DeclareMathOperator{\Fun}{Fun}          % The set of all functions from one set to another
	\DeclareMathOperator{\Perm}{Perm}        % The set of all permutations on a set
	\newcommand{\oplusbar}{\mathbin{\ooalign{$\hidewidth\overline{\oplus}\hidewidth$\cr$\phantom{\oplus}$}}}
	\newcommand{\otimesbar}{\mathbin{\ooalign{$\hidewidth\overline{\otimes}\hidewidth$\cr$\phantom{\otimes}$}}}
	
	%---The following code defines the title, author, and date of the document.
	\title{Practice proofs \#1}
	\author{Anonymous}
	\date{\today}   % Using \today automatically updates to the document's build date
%----------------------------------
%---------IF YOU WANT TO DEFINE YOUR OWN MACROS, YOU CAN DO SO FROM HERE ...

%---------... TO HERE

\begin{document}
\maketitle
\thispagestyle{fancy}
%-----------THIS IS WHERE THE MAIN DOCUMENT BEGINS.

%-----------EDIT FROM HERE

\begin{defn} \label{defn:Throdd}
	An integer $n$ is called \emph{threven} if it can be written in the form $n=3k$ for another integer $k$.
	If $n=3k+1$ for another integer $k$, then $n$ is called \emph{throdd}.
	If $n=3k+2$ for integer $k$, then $n$ is called \emph{throdder}.
\end{defn}

For these proofs, you may assume that every integer is precisely one of threven, throdd or throdder.

\section*{Proof 1}

\begin{thm} \label{thm:ThroddSquares}
	There are no throdder squares.
\end{thm}


\begin{proof}
	From \textbf{Definition} \ref{defn:Throdd}, every integer $n$ is either only threven, throdd, or throdder. When $n$ is threven, it can be expressed in the form $n = 3k$ for another integer $k$.
	The square of $n$ is found by taking $n$ multiplied by itself. Thus, when $n$ is threven, the square of $n$ can be expressed as 
	
	\begin{equation}
	\begin{aligned}
	n^2 &= (3k) * (3k)\\			
		&= 9k^2\\
		&= 3(3k^2) = 3j \textrm{, where $j$ is another integer.}
	\end{aligned}
	\end{equation}
	
	Since when $n$ is threven $n^2$ can be re-expressed in the form $3j$ where j is another integer, $n^2$ is also threven.\\
	
	Next, when $n$ is throdd, it can be expressed in the form $n = 3k + 1$ for another integer $k$. Thus, the square of $n$ when it is throdd is 
	
	\begin{equation}
	\begin{aligned}
	n^2 &= (3k+1) * (3k+1)\\
		&= 9k^2 + 6k + 1\\
		&= 3(3k^2+2k) + 1 = 3j + 1 \textrm{, where $j$ is another integer.}		
	\end{aligned}
	\end{equation}
	
	Since $n^2$ can be re-expressed in the form $3j+1$ where j is another integer, $n^2$ is throdd when $n$ is throdd.\\
	
	
	Finally, when $n$ is throdder, it can be expressed in the form $n = 3k + 2$ for another integer $k$. Hence, its square is
	
	\begin{equation}
	\begin{aligned}
	n^2 &= (3k+2) * (3k+2)\\
		&= 9k^2 + 12k + 4\\
		&= 3(3k^2 + 4k + 1) + 1 = 3j + 1 \textrm{, where $j$ is another integer.}
	\end{aligned}
	\end{equation}
	
	Since $n^2$ can be re-expressed in the form $3j+1$ where j is another integer, $n^2$ is throdd when $n$ is throdder.\\
	
	From \textbf{Definition} \ref{defn:Throdd} every integer $n$ is precisely one of threven, throdd, or throdder. Therefore, because $n^2$ is never throdder when $n$ is threven, throdd, or throdder, there are no throdder squares.
\end{proof}

\section*{Proof 2}

\begin{lem} \label{lem:3}
	There is a positive real number $z$ such that $z^2 = 3$.
\end{lem}
\begin{proof}
	Consider a right triangle with a base side of length 1, and an angle of $60^{\circ}$. We know that $\sin{\frac{\pi}{3}} = \frac{\sqrt{3}}{2}}$. Thus, the length of the hypothenus is $\sqrt{3}$. 
	Any length must be a positive real number, so there is a positive real number $z$ such that $z^2=3$.
\end{proof}

\begin{thm} \label{thm:Root3Irrational}
	There is a positive irrational number $x$ for which $x^2=3$.
\end{thm}


\begin{proof}
	(By contradiction). Suppose there were a rational number whose square was $3$. Such a rational number could be expressed as $\frac{a}{b}$ for integers $a$ and $b$. Furthermore, by reducing the fraction, 
	we can assume that $a$ and $b$ are $\emph{not both}$ even, since if they were the fraction could be further reduced. This can also be expressed as the fact that the greatest common divisor of $a$ and $b$ is $1$.\\
	
	Thus, we can express $\sqrt{3}$ and its square as such
	
	\begin{equation}
	\begin{aligned}
	\sqrt{3} &= \frac{a}{b}\\
	3 &= \frac{a^2}{b^2} \quad \textrm{(Squaring both sides)}\\
	\end{aligned}
	\end{equation}
	
	Re-arranging $3 = \frac{a^2}{b^2}$, we get
	
	\begin{equation} \label{eq:5}
	\begin{aligned}
	a^2 &= 3b^2\\
	\end{aligned}
	\end{equation}
	
	This shows us that $a^2$ is a multiple of $3$, since it can be divided by $3$. Hence, we can write $a = 3c$ where $c$ is an integer and substitute it into equation \ref{eq:5}. This gives us
	
	\begin{equation}
	\begin{aligned}
	(3c)^2 &= 3b^2\\
	9c^2 &= 3b^2 \\
	3c^2 &= b^2\\
	b^2 &= 3c^2
	\end{aligned}
	\end{equation}
	
	Thus, we see that $b^2$ is \emph{also} a multiple of $3$.\\


	Since $a^2$ is a multiple of $3$, we know that $a$ is also a multiple of $3$. Similarly, since $b^2$ is a multiple of $3$, $b$ is also a multiple of $3$.
	Consequently, $\frac{a}{b}$ can still be further reduced, contradicting the	reducedness of fraction $\frac{a}{b}$. Hence, there cannot be a rational number whose square is $3$.
	By lemma \ref{lem:3}, there is a positive real number whose square is $3$. So there is a positive irrational number $x$ for which $x^2=3$.
	
\end{proof}


\end{document}



