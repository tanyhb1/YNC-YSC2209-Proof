%---------DO NOT EDIT THIS INDENTED SECTION
	% Preamble
	\documentclass[11pt,reqno,oneside,a4paper]{article}
	\usepackage[a4paper,includeheadfoot,left=35mm,right=35mm,top=00mm,bottom=30mm,headheight=40mm]{geometry} %sets up the margins
	\input{../texHead-Proof-Standard} % Use the standard texHead for this module. You should not edit this file.
	\input{../texHead-Proof-Personal} % Use your personal additional macros. You may edit this file.
	%---The following code defines the title, author, and date of the document.
	\title{Practice proofs \#4}
	\author{Anonymous}
	\date{\today}   % Using \today automatically updates to the document's build date
%----------------------------------
%---------IF YOU WANT TO DEFINE YOUR OWN MACROS, YOU CAN DO SO EITHER IN ../texHead-Proof-Personal.tex OR FROM HERE ...

%---------... TO HERE
\begin{document}
\maketitle
\thispagestyle{fancy}

%-----------EDIT FROM HERE


We explicitly construct a bijection mapping from the unit interval to the whole real line.

\begin{thm} \label{thm:BijectionRInterval}
	Define $f : (0,1) \to \RR$ by
	$$
		f(x) =\frac{1}{1-x} - \frac{1}{x}.
	$$
	The function $f$ is a bijection.
\end{thm}

\begin{proof}
	To prove that the function $f$ is a bijection, we will first prove that $f$ is an injection and a surjection.\\
	
	\noindent\emph{Injection}. Consider $f(x) =\frac{1}{1-x} - \frac{1}{x}$. Then, $f'(x) = \frac{1}{(1-x)^2} + \frac{1}{x^2}$. Since $(1-x)^2$ and $x^2$ are always positive for all values of $0 < x < 1$,
	we know that $f'(x) > 0$ for all $0 < x < 1$. Thus, $f(x) = \frac{1}{1-x} - \frac{1}{x}$ is strictly increasing. Since $f$ is strictly increasing, we know that when $x \neq y$, $f(x) \neq f(y)$. Therefore, $f$ is an 
	injective function.\\
	
	\noindent\emph{Surjection}. Consider $f(1-\frac{1}{M})$ and $f(\frac{1}{M})$ for large $M > 0$. Then, 
	
	\begin{align*}
	f(1-\frac{1}{M}) &= \frac{1}{1-(1-\frac{1}{M})} - \frac{1}{1-\frac{1}{M}} & f(\frac{1}{M}) &= \frac{1}{1-\frac{1}{M}} - \frac{1}{\frac{1}{M}}\\
					&= \frac{1}{\frac{1}{M}} - \frac{1}{\frac{M-1}{M}} & 					&= \frac{M}{M-1} - M\\
					&= M - \frac{M}{M-1} & &= -(M-\frac{M}{M-1}) = -f(1-\frac{1}{M})
	\end{align*}
	
	\noindent For large $M > 0$, we observe that $\frac{M}{M-1} \in \mathbb{R}$ and tends to 1 as $M$ increases. Thus, for $f(1-\frac{1}{M})$, $M - \frac{M}{M-1} \in \mathbb{R}$ and
	$f(1-\frac{1}{M})$ tends to $M - 1$ as $M$ increases.	Since we note that $f(\frac{1}{M}) = -(M-\frac{M}{M-1}) = -f(1-\frac{1}{M})$, $f(\frac{1}{M})$ tends towards $-(M - 1)$. 
	By using a sufficiently large value of $M > 0$, we know that for every $y \in \mathbb{R}$, $-(M-1) < y < M-1$ since we can always increase $M$ to include a larger magnitude of $y$.
	Therefore, $f(\frac{1}{M}) < y < f(1-\frac{1}{M})$ for a sufficiently large $M > 0$, and by the intermediate value theorem we know that there exists a $\frac{1}{M} < x < 1-\frac{1}{M}}$ for every 
	$y\in\mathbb{R}$ where $M>0$ is sufficiently large.\\
	
	\noindent Thus, for every $y \in \mathbb{R}$, there exists some $M>0$ such that $f(\frac{1}{M}) < y < f(1-\frac{1}{M})$. When $x = \frac{1}{M}$ and $x = (1-\frac{1}{M})$, we know that $0<x<1$. Hence,
	$f$ is a surjective function since for each $y\in\mathbb{R}$, $y=f(x)$ for some $0<x<1$.\\
	
	\noindent\emph{Bijection}. Since $f$ is an injective function and a surjective function, $f$ is a bijection.
\end{proof}

\end{document}



