%---------DO NOT EDIT THIS INDENTED SECTION
	% Preamble
	\documentclass[11pt,reqno,oneside,a4paper]{article}
	\usepackage[a4paper,includeheadfoot,left=35mm,right=35mm,top=00mm,bottom=30mm,headheight=40mm]{geometry} %sets up the margins
	\input{../texHead-Proof-Standard} % Use the standard texHead for this module. You should not edit this file.
	\input{../texHead-Proof-Personal} % Use your personal additional macros. You may edit this file.
	%---The following code defines the title, author, and date of the document.
	\title{Practice proofs \#5}
	\author{Anonymous}
	\date{\today}   % Using \today automatically updates to the document's build date
%----------------------------------
%---------IF YOU WANT TO DEFINE YOUR OWN MACROS, YOU CAN DO SO EITHER IN ../texHead-Proof-Personal.tex OR FROM HERE ...

%---------... TO HERE
\begin{document}
\maketitle
\thispagestyle{fancy}

%-----------EDIT FROM HERE

We study how cardinality interacts with subsets and power sets.

Recall that we say that two sets are equinumerous if they have the same cardinality.
The following proposition says that the operation ``take the power set'' preserves equinumerosity.
	
\begin{prop} \label{prop:PowerSetsPreserveEquinumerosity}
	Suppose $\# X=\# Y$.
	Then $\# \power(X)=\# \power(Y)$.
	Moreover, if $\# X\leq\# Y$, then $\# \power(X)\leq\# \power(Y)$.
\end{prop}

\begin{proof}
	Let $f:x\rightarrow y$ and $g:\power(X)\rightarrow \power(Y)$.
	If $\#X \leq \#Y$, we know that $f$ is injective. We define $g$ by $g(A) = \{ f(x) : x\in A \}$, where $A\subset \power(X)$. For $A,B \subset \power(X)$, 
	\begin{alignat}{2}
	&&g(A) &= g(B)\notag\\
	\implies\quad&& \{ f(x): x\in A \} &= \{ f(x) : x \in B \}\notag\\
	\implies\quad&& A &= B\notag
	\end{alignat}
	Thus, since $g(A) = g(B)$ if and only if $A = B$, we know that $g$ is injective when $f$ is injective. Hence, if $\#X\leq \#Y$, $g$ is injective and therefore $\# \power(X) \leq \# \power(Y)$.\\
	
	If we suppose that $\#X = \#Y$, then $f$ is bijective. To show $\# \power(X) = \# \power(Y)$, we have to show that $g$ is not only injective as shown above but also surjective. We know $g$ is surjective 
	if and only if $\forall Z \subset \power(Y), \exists A \subset \power(X)$ such that $g(A) = Z$. Hence, 
	\begin{alignat}{2}
	&&g(A) &= Z\notag\\
	\implies\quad && \{f(x) : x \in A \} &= Z\notag\\
	\implies\quad && A &= \{ f^{-1}(x) : x \in Z \}\notag
	\end{alignat}
	
	where $f^{-1}$ exists and is a bijective function because $f$ is a bijective function. Thus, $g$ is surjective when $f$ is bijective. Because $g$ is both injective and surjective and thus bijective when $f$ is bijective, we know that 
	$\# \power(X) = \# \power(Y)$ when $\#X = \#Y$.
\end{proof}

\begin{prop} \label{prop:SubsetCardinality}
	If $X\subset Y$, then $\# X\leq \# Y$.
\end{prop}

\begin{proof}
	Let $f : X \rightarrow Y$. Define $f$ by $f(x) = x$, where $x\in X$ and $f(x)=x \in Y$ since $X\subset Y$. Thus, for $a,b \in X$, 	
	\begin{alignat}{2}
	&&f(a) &= f(b) \notag\\
	\implies\quad&& a &= b \notag
	\end{alignat}
	
	and we know $f$ is injective since $f(a) = f(b)$ if and only if $a = b$. Since $f$ is injective when $X \subset Y$, $\# X\leq \# Y$.
\end{proof}

\begin{cor} \label{cor:IntersectionCardinality}
	$\# (X\cap Y) \leq \# X$.
\end{cor}

\begin{proof}
	Let $S = X\cap Y$. We know that $S \subset X$ and $S \subset Y$. Thus, from Proposition 2, since $S \subset X$, $ \# S \leq \# X$, and therefore $\# (X \cap Y) \leq \# X$.
\end{proof}

\end{document}



