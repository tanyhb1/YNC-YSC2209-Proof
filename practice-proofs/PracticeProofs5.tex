%---------DO NOT EDIT THIS INDENTED SECTION
	% Preamble
	\documentclass[11pt,reqno,oneside,a4paper]{article}
	\usepackage[a4paper,includeheadfoot,left=35mm,right=35mm,top=00mm,bottom=30mm,headheight=40mm]{geometry} %sets up the margins
	%%%%%%%%%%%%%%%%%%%%%%%%%%%%%%%%%%%%%%%%%%%%%%%%%%%%%%%%%%%%%%%%%%%%%%%%%%%%%%%%
%
% This file contains some standard modifications to basic LaTeX2e and
% the article documentclass. DO NOT EDIT THIS FILE, but do look through
% and make use of the shorthands defined herein.
%
%%%%%%%%%%%%%%%%%%%%%%%%%%%%%%%%%%%%%%%%%%%%%%%%%%%%%%%%%%%%%%%%%%%%%%%%%%%%%%%%

% Standard packages
\usepackage{amssymb,amsmath,amsthm}
\usepackage{xcolor,graphicx}
\usepackage{verbatim}
\usepackage{hyperref}
% Layout of headers & footers
\usepackage{titling}
\usepackage{fancyhdr}
\pagestyle{fancy} \lhead{{\theauthor}} \chead{} \rhead{} \lfoot{} \cfoot{\thepage} \rfoot{}

% Hyphenation
\hyphenation{non-zero}

% Instroctor's email address
\newcommand{\InstEmail}{dave.smith@yale-nus.edu.sg}

% Theorem definitions in the amsthm standard
\newtheorem{thm}{Theorem}
\newtheorem{lem}[thm]{Lemma}
\newtheorem{sublem}[thm]{Sublemma}
\newtheorem{prop}[thm]{Proposition}
\newtheorem{cor}[thm]{Corollary}
\newtheorem{conc}[thm]{Conclusion}
\newtheorem{conj}[thm]{Conjecture}
\theoremstyle{definition}
\newtheorem{defn}[thm]{Definition}
\newtheorem{cond}[thm]{Condition}
\newtheorem{asm}[thm]{Assumption}
\newtheorem{ntn}[thm]{Notation}
\newtheorem{prob}[thm]{Problem}
\theoremstyle{remark}
\newtheorem{rmk}[thm]{Remark}
\newtheorem{eg}[thm]{Example}
\newtheorem*{hint}{Hint}

%% Mathmode shortcuts
% Number sets
\newcommand{\NN}{\mathbb N}              % The set of naturals
\newcommand{\NNzero}{\NN^0}              % The set of naturals including zero
\newcommand{\NNone}{\NN}                 % The set of naturals excluding zero
\newcommand{\ZZ}{\mathbb Z}              % The set of integers
\newcommand{\QQ}{\mathbb Q}              % The set of rationals
\newcommand{\RR}{\mathbb R}              % The set of reals
\newcommand{\CC}{\mathbb C}              % The set of complex numbers
\newcommand{\KK}{\mathbb C}              % An arbitrary field
% Modern typesetting for the real and imaginary parts of a complex number
\renewcommand{\Re}{\operatorname*{Re}} \renewcommand{\Im}{\operatorname*{Im}}
% Upright d for derivatives
\newcommand{\D}{\ensuremath{\,\mathrm{d}}}
% Make epsilons look more different from the element symbol
\renewcommand{\epsilon}{\varepsilon}
% Always use slanted forms of \leq, \geq
\renewcommand{\geq}{\geqslant} \renewcommand{\leq}{\leqslant}
% Shorthand for some relations
\newcommand{\po}{\preceq} \newcommand{\rel}{{\mathcal R}} \newcommand{\rels}{\mathbin{\scriptstyle{\mathcal R}}}
% Shorthand for "if and only if" symbol
\newcommand{\Iff}{\ensuremath{\Leftrightarrow}}
% Make bold symbols for vectors
\providecommand{\BVec}[1]{\mathbf{#1}}
% Barred forms of \oplus and \otimes to represent the descents of these binary operators
\newcommand{\oplusbar}{\mathbin{\ooalign{$\hidewidth\overline{\oplus}\hidewidth$\cr$\phantom{\oplus}$}}} \newcommand{\otimesbar}{\mathbin{\ooalign{$\hidewidth\overline{\otimes}\hidewidth$\cr$\phantom{\otimes}$}}}
% Mathematical operators used in Proof
\DeclareMathOperator{\sgn}{sgn}          % The signum of a real number
\DeclareMathOperator{\power}{\mathcal{P}} % The power set of a set
\DeclareMathOperator{\Id}{Id}            % The identity function
\DeclareMathOperator{\Fun}{Fun}          % The set of functions from one set to another
\DeclareMathOperator{\Perm}{Perm}        % The set of permutations on a set
\DeclareMathOperator{\GCD}{GCD}          % The greatest common divisor of two integers
\newcommand{\abs}[1]{\left\lvert#1\right\rvert} % The absolute value of a real number or modulus of a complex number, with automatically scaling delimiters
 % Use the standard texHead for this module. You should not edit this file.
	%%%%%%%%%%%%%%%%%%%%%%%%%%%%%%%%%%%%%%%%%%%%%%%%%%%%%%%%%%%%%%%%%%%%%%%%%%%%%%%%
%
% You can make any edits you like to this file. It is designed as a place
% for you to define your own macros that you want to use across many
% LaTeX documents. You can also define macros in a single document,
% but if you find yourself reusing the same macro in several documents,
% then it probably belongs in here. Look in the "%% Mathmode shortcuts"
% section of texHead-Proof-Standard.tex for some examples of how to define
% your own macros.
%
%%%%%%%%%%%%%%%%%%%%%%%%%%%%%%%%%%%%%%%%%%%%%%%%%%%%%%%%%%%%%%%%%%%%%%%%%%%%%%%%

% My own macros
\newcommand{\ominusbar}{\mathbin{\ooalign{$\hidewidth\overline{\ominus}\hidewidth$\cr$\phantom{\ominus}$}}}

 % Use your personal additional macros. You may edit this file.
	%---The following code defines the title, author, and date of the document.
	\title{Practice proofs \#5}
	\author{Anonymous}
	\date{\today}   % Using \today automatically updates to the document's build date
%----------------------------------
%---------IF YOU WANT TO DEFINE YOUR OWN MACROS, YOU CAN DO SO EITHER IN ../texHead-Proof-Personal.tex OR FROM HERE ...

%---------... TO HERE
\begin{document}
\maketitle
\thispagestyle{fancy}

%-----------EDIT FROM HERE

We study how cardinality interacts with subsets and power sets.

Recall that we say that two sets are equinumerous if they have the same cardinality.
The following proposition says that the operation ``take the power set'' preserves equinumerosity.
	
\begin{prop} \label{prop:PowerSetsPreserveEquinumerosity}
	Suppose $\# X=\# Y$.
	Then $\# \power(X)=\# \power(Y)$.
	Moreover, if $\# X\leq\# Y$, then $\# \power(X)\leq\# \power(Y)$.
\end{prop}

\begin{proof}
	Let $f:x\rightarrow y$ and $g:\power(X)\rightarrow \power(Y)$.
	If $\#X \leq \#Y$, we know that $f$ is injective. We define $g$ by $g(A) = \{ f(x) : x\in A \}$, where $A\subset \power(X)$. For $A,B \subset \power(X)$, 
	\begin{alignat}{2}
	&&g(A) &= g(B)\notag\\
	\implies\quad&& \{ f(x): x\in A \} &= \{ f(x) : x \in B \}\notag\\
	\implies\quad&& A &= B\notag
	\end{alignat}
	Thus, since $g(A) = g(B)$ if and only if $A = B$, we know that $g$ is injective when $f$ is injective. Hence, if $\#X\leq \#Y$, $g$ is injective and therefore $\# \power(X) \leq \# \power(Y)$.\\
	
	If we suppose that $\#X = \#Y$, then $f$ is bijective. To show $\# \power(X) = \# \power(Y)$, we have to show that $g$ is not only injective as shown above but also surjective. We know $g$ is surjective 
	if and only if $\forall Z \subset \power(Y), \exists A \subset \power(X)$ such that $g(A) = Z$. Hence, 
	\begin{alignat}{2}
	&&g(A) &= Z\notag\\
	\implies\quad && \{f(x) : x \in A \} &= Z\notag\\
	\implies\quad && A &= \{ f^{-1}(x) : x \in Z \}\notag
	\end{alignat}
	
	where $f^{-1}$ exists and is a bijective function because $f$ is a bijective function. Thus, $g$ is surjective when $f$ is bijective. Because $g$ is both injective and surjective and thus bijective when $f$ is bijective, we know that 
	$\# \power(X) = \# \power(Y)$ when $\#X = \#Y$.
\end{proof}

\begin{prop} \label{prop:SubsetCardinality}
	If $X\subset Y$, then $\# X\leq \# Y$.
\end{prop}

\begin{proof}
	Let $f : X \rightarrow Y$. Define $f$ by $f(x) = x$, where $x\in X$ and $f(x)=x \in Y$ since $X\subset Y$. Thus, for $a,b \in X$, 	
	\begin{alignat}{2}
	&&f(a) &= f(b) \notag\\
	\implies\quad&& a &= b \notag
	\end{alignat}
	
	and we know $f$ is injective since $f(a) = f(b)$ if and only if $a = b$. Since $f$ is injective when $X \subset Y$, $\# X\leq \# Y$.
\end{proof}

\begin{cor} \label{cor:IntersectionCardinality}
	$\# (X\cap Y) \leq \# X$.
\end{cor}

\begin{proof}
	Let $S = X\cap Y$. We know that $S \subset X$ and $S \subset Y$. Thus, from Proposition 2, since $S \subset X$, $ \# S \leq \# X$, and therefore $\# (X \cap Y) \leq \# X$.
\end{proof}

\end{document}



