%---------DO NOT EDIT THIS INDENTED SECTION
	% Preamble
	\documentclass[11pt,reqno,oneside,a4paper]{article}
	\usepackage[a4paper,includeheadfoot,left=35mm,right=35mm,top=00mm,bottom=30mm,headheight=40mm]{geometry} %sets up the margins
	\input{../texHead-Proof-Standard} % Use the standard texHead for this module. You should not edit this file.
	\input{../texHead-Proof-Personal} % Use your personal additional macros. You may edit this file.
	%---The following code defines the title, author, and date of the document.
	\title{Practice proofs \#7}
	\author{Anonymous}
	\date{\today}   % Using \today automatically updates to the document's build date
%----------------------------------
%---------IF YOU WANT TO DEFINE YOUR OWN MACROS, YOU CAN DO SO EITHER IN ../texHead-Proof-Personal.tex OR FROM HERE ...

%---------... TO HERE
\begin{document}
\maketitle
\thispagestyle{fancy}

%-----------EDIT FROM HERE


Here we take for granted the set of integers, and all familiar properties of addition, subtraction, and multiplication of integers.
We also take for granted the ``integral domain'' property of integers, which states that $xy = 0$ implies $x = 0$ or $y = 0$, for all integers $x,y$.
Equivalently, if $x \neq 0$ and $y \neq 0$ then $xy \neq 0$.
This implies the cancellation property: if $x,y,z \in \ZZ$, and $z \neq 0$, and $xz = yz$, then $x = y$.

A \emph{fraction} will mean an ordered pair $(x,y) \in \ZZ^2$ such that $y \neq 0$. Let $F$ be the set of fractions.
Consider the relation $\sim$ on $F$ given by the rule
$$
	(x,y) \sim (u,v) \text{ means that } xv = yu.
$$
The point of $\sim$ is that if two fractions obey $(x,y)\sim(u,v)$, then they are two expressions of the same rational number.
For example, $(2,4)\sim(12,24)$, and they both correspond to the rational number that we are used to representing $\frac{1}{2}$.

\begin{thm} \label{thm:EquivaentFractions}
	The relation $\sim$ is an equivalence relation on the set $F$.
\end{thm}

\begin{proof}
	I will show that the relation $\sim$ is an equivalence relation on the set $F$ by demonstrating that $\sim$ satisfies the three properties of an equivalence relation.\\
	
	\noindent\emph{Reflexivity}. $\forall a \in F$, where $a=(x,y) \in \mathbb{Z}^2$ such that $x,y \in \mathbb{Z}$ and $y \neq 0$,
	\begin{alignat}{2}
	&& a &\sim a \notag\\
	\implies\quad&& (x,y) &\sim (x,y)\notag\\
	\implies\quad&& xy &= xy \notag
	\end{alignat}	
	\noindent Thus, by definition of reflexivity, we know that the relation $\sim$ is reflexive.\\
	
	\noindent\emph{Symmetry}. $\forall a,b \in F$, where $a = (a_x, a_y) \in \mathbb{Z}^2$ and $b = (b_x, b_y) \in \mathbb{Z}^2$,	
	\begin{alignat}{2}
	&&a &\sim b \qquad&&\qquad &b &\sim a &\notag\\
	\implies\quad&& (a_x, a_y) &\sim (b_x, b_y) \qquad&&\implies\quad&(b_x, b_y) &\sim (a_x, a_y) &\notag\\
	\implies\quad&& a_x(b_y) &= a_y(b_x)  \qquad&&\implies\quad&b_x(a_y) &= b_y(a_x) &\notag\\ 
	\implies\quad&& b_x(a_y) &= b_y(a_x)  \qquad&&\implies\quad&a_x(b_y) &= a_y(b_x) &\notag\\
	\implies\quad&& b &\sim a \qquad&&\implies\quad& a&\sim b&\notag\\\notag
	\end{alignat}
	
	\noindent Thus, by definition of symmetry, the relation $\sim$ is symmetric since $\forall a,b \in F$, $a\sim b$ if and only if $b\sim a$.\\
	
	\noindent\emph{Transitivity}. $\forall a,b,c \in F$, suppose $a \sim b$ and $b \sim c$. Then we know 
	\begin{alignat}{2}
	&&a &\sim b \qquad&&\qquad &b&\sim c &&\notag\\
	\implies\quad&& (a_x,a_y) &\sim (b_x, b_y) \qquad&&\implies\quad &(b_x, b_y) &\sim (c_x, c_y)&&\notag\\
	\implies\quad&& a_x(b_y) &= a_y(b_x) \qquad&&\implies\quad &b_x(c_y) &= b_y(c_x)\notag
	\end{alignat}

\noindent Multiplying $a_x(b_y) = a_y(b_x)$ by $b_x(c_y)$,  	
	\begin{alignat}{2}
	&& a_x(b_y) &= a_y(b_x)\notag\\
	\implies\quad&& a_x(b_y)\times b_x(c_y) &= a_y(b_x) \times b_x(c_y)\notag\\
	\implies\quad&& a_x(b_y)\times b_x(c_y) &= a_y(b_x) \times b_y(c_x)\notag\\
	\implies\quad&& a_x(c_y) \times b_x(b_y) &= a_y(c_x) \times b_x(b_y)\notag
	\end{alignat}
	
\noindent By definition of fractions, we know that $a_y, b_y, c_y \neq 0$. Consequently, we can apply the cancellation property to cancel out $b_x(b_y)$: 
	\begin{alignat}{2}
	&& a_x(c_y) \times b_x(b_y) &= a_y(c_x) \times b_x(b_y)\notag\\
	\implies\quad&& a_x(c_y) & = a_y(c_x) \notag \\
	\implies\quad&& a &\sim c \notag
	\end{alignat}
	
	\noindent Thus, since $a \sim b$ and $b \sim c$ implies $a \sim c$, the relation $\sim$ is transitive. Because the relation $\sim$ on set $F$ satisfies the three properties of an equivalence relation, $\sim$ is an equivalence relation on the set $F$.
\end{proof}

\end{document}



