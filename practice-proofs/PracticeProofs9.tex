%---------DO NOT EDIT THIS INDENTED SECTION
	% Preamble
	\documentclass[11pt,reqno,oneside,a4paper]{article}
	\usepackage[a4paper,includeheadfoot,left=35mm,right=35mm,top=00mm,bottom=30mm,headheight=40mm]{geometry} %sets up the margins
	%%%%%%%%%%%%%%%%%%%%%%%%%%%%%%%%%%%%%%%%%%%%%%%%%%%%%%%%%%%%%%%%%%%%%%%%%%%%%%%%
%
% This file contains some standard modifications to basic LaTeX2e and
% the article documentclass. DO NOT EDIT THIS FILE, but do look through
% and make use of the shorthands defined herein.
%
%%%%%%%%%%%%%%%%%%%%%%%%%%%%%%%%%%%%%%%%%%%%%%%%%%%%%%%%%%%%%%%%%%%%%%%%%%%%%%%%

% Standard packages
\usepackage{amssymb,amsmath,amsthm}
\usepackage{xcolor,graphicx}
\usepackage{verbatim}
\usepackage{hyperref}
% Layout of headers & footers
\usepackage{titling}
\usepackage{fancyhdr}
\pagestyle{fancy} \lhead{{\theauthor}} \chead{} \rhead{} \lfoot{} \cfoot{\thepage} \rfoot{}

% Hyphenation
\hyphenation{non-zero}

% Instroctor's email address
\newcommand{\InstEmail}{dave.smith@yale-nus.edu.sg}

% Theorem definitions in the amsthm standard
\newtheorem{thm}{Theorem}
\newtheorem{lem}[thm]{Lemma}
\newtheorem{sublem}[thm]{Sublemma}
\newtheorem{prop}[thm]{Proposition}
\newtheorem{cor}[thm]{Corollary}
\newtheorem{conc}[thm]{Conclusion}
\newtheorem{conj}[thm]{Conjecture}
\theoremstyle{definition}
\newtheorem{defn}[thm]{Definition}
\newtheorem{cond}[thm]{Condition}
\newtheorem{asm}[thm]{Assumption}
\newtheorem{ntn}[thm]{Notation}
\newtheorem{prob}[thm]{Problem}
\theoremstyle{remark}
\newtheorem{rmk}[thm]{Remark}
\newtheorem{eg}[thm]{Example}
\newtheorem*{hint}{Hint}

%% Mathmode shortcuts
% Number sets
\newcommand{\NN}{\mathbb N}              % The set of naturals
\newcommand{\NNzero}{\NN^0}              % The set of naturals including zero
\newcommand{\NNone}{\NN}                 % The set of naturals excluding zero
\newcommand{\ZZ}{\mathbb Z}              % The set of integers
\newcommand{\QQ}{\mathbb Q}              % The set of rationals
\newcommand{\RR}{\mathbb R}              % The set of reals
\newcommand{\CC}{\mathbb C}              % The set of complex numbers
\newcommand{\KK}{\mathbb C}              % An arbitrary field
% Modern typesetting for the real and imaginary parts of a complex number
\renewcommand{\Re}{\operatorname*{Re}} \renewcommand{\Im}{\operatorname*{Im}}
% Upright d for derivatives
\newcommand{\D}{\ensuremath{\,\mathrm{d}}}
% Make epsilons look more different from the element symbol
\renewcommand{\epsilon}{\varepsilon}
% Always use slanted forms of \leq, \geq
\renewcommand{\geq}{\geqslant} \renewcommand{\leq}{\leqslant}
% Shorthand for some relations
\newcommand{\po}{\preceq} \newcommand{\rel}{{\mathcal R}} \newcommand{\rels}{\mathbin{\scriptstyle{\mathcal R}}}
% Shorthand for "if and only if" symbol
\newcommand{\Iff}{\ensuremath{\Leftrightarrow}}
% Make bold symbols for vectors
\providecommand{\BVec}[1]{\mathbf{#1}}
% Barred forms of \oplus and \otimes to represent the descents of these binary operators
\newcommand{\oplusbar}{\mathbin{\ooalign{$\hidewidth\overline{\oplus}\hidewidth$\cr$\phantom{\oplus}$}}} \newcommand{\otimesbar}{\mathbin{\ooalign{$\hidewidth\overline{\otimes}\hidewidth$\cr$\phantom{\otimes}$}}}
% Mathematical operators used in Proof
\DeclareMathOperator{\sgn}{sgn}          % The signum of a real number
\DeclareMathOperator{\power}{\mathcal{P}} % The power set of a set
\DeclareMathOperator{\Id}{Id}            % The identity function
\DeclareMathOperator{\Fun}{Fun}          % The set of functions from one set to another
\DeclareMathOperator{\Perm}{Perm}        % The set of permutations on a set
\DeclareMathOperator{\GCD}{GCD}          % The greatest common divisor of two integers
\newcommand{\abs}[1]{\left\lvert#1\right\rvert} % The absolute value of a real number or modulus of a complex number, with automatically scaling delimiters
 % Use the standard texHead for this module. You should not edit this file.
	%%%%%%%%%%%%%%%%%%%%%%%%%%%%%%%%%%%%%%%%%%%%%%%%%%%%%%%%%%%%%%%%%%%%%%%%%%%%%%%%
%
% You can make any edits you like to this file. It is designed as a place
% for you to define your own macros that you want to use across many
% LaTeX documents. You can also define macros in a single document,
% but if you find yourself reusing the same macro in several documents,
% then it probably belongs in here. Look in the "%% Mathmode shortcuts"
% section of texHead-Proof-Standard.tex for some examples of how to define
% your own macros.
%
%%%%%%%%%%%%%%%%%%%%%%%%%%%%%%%%%%%%%%%%%%%%%%%%%%%%%%%%%%%%%%%%%%%%%%%%%%%%%%%%

% My own macros
\newcommand{\ominusbar}{\mathbin{\ooalign{$\hidewidth\overline{\ominus}\hidewidth$\cr$\phantom{\ominus}$}}}

 % Use your personal additional macros. You may edit this file.
	%---The following code defines the title, author, and date of the document.
	\title{Practice proofs \#9}
	\author{Anonymous}
	\date{\today}   % Using \today automatically updates to the document's build date
%----------------------------------
%---------IF YOU WANT TO DEFINE YOUR OWN MACROS, YOU CAN DO SO EITHER IN ../texHead-Proof-Personal.tex OR FROM HERE ...

%---------... TO HERE
\begin{document}
\maketitle
\thispagestyle{fancy}

%-----------EDIT FROM HERE

\begin{abstract}
	%---------DO NOT EDIT THIS FILE

The following exercise is meant to be challenging.
It is expected that your submission will not be perfect, but you will receive feedback on your submission so that you can improve it.
You \textbf{must} complete this exercise entirely on your own.
Do \textbf{not} collaborate with your peers on this exercise.
Do \textbf{not} ask the tutors for help, or search books / web for assistance.
There are hints in the .tex file.
There are more hints in the document \href{run:./PracticeProofsExtraHints.pdf}{PracticeProofsExtraHints.pdf}.
If you need help, please contact Dave by email: \href{mailto:\InstEmail}{\nolinkurl{\InstEmail}}.

You will be graded on participation.
That means it is OK to not have perfect answers, but you will get more out of the exercise if you try your best.

Be sure not to put your name anywhere on the document.
{\color{red}Before submitting, please delete the begin \& end abstract tags and everything between them.}



	
	I fitted the whole model solutions onto 2 pages.
\end{abstract}

We argue that real multiplication, as defined through multiplication of rational Cauchy sequences, satisfies the corresponding five field axioms.

\begin{lem} \label{lem:ExistsNonzeroCauchySequence}
	Suppose that $x\in\RR\setminus\{0_\RR\}$.
	Then there is a rational $M>0$ and a Cauchy sequence $(x_n)_{n\in\NN}$ such every term has $|x_n|>M$ and $(x_n)_{n\in\NN}$ belongs to the equivalence class $x$.
\end{lem}

\begin{proof}
	%--------------EDIT THIS PART--------------%
	% Hint: The idea is to argue that there is a rational Cauchy sequence in $x$ for which at most finitely many terms are less than $M$.
	% Try contradiction; you should end up proving $x=0_\RR$.
	% After that, you can fix the finitely many ``problem'' terms without changing the equivalence class of your Cauchy sequence.
	%------------------------------------------%
\end{proof}

\begin{thm} \label{thm:RealMultiplication}
	Real multiplication, defined as the descent of multiplication of rational Cauchy sequences, obeys field axioms~\textup{(6)--(10)}.
\end{thm}

\begin{proof}
	%--------------EDIT THIS PART--------------%
	% Hint: Axioms~(6)--(9) essentially follow from the corresponding properties for rational numbers.
	% For axiom~(10), the tricky part is showing that the sequence you want to represent the multiplicative inverse is actually a Cauchy sequence.
	% It might be easiest to start by structuring the proof as if you have a magical lemma that guarantees the sequence is Cauchy, then go back and fill the gap, using lemma~\ref{lem:ExistsNonzeroCauchySequence} and some ingenuity.
	%------------------------------------------%
\end{proof}

\end{document}



