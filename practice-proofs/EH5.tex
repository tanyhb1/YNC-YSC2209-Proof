\subsection*{Cardinality and subsets}

\begin{prop} \label{prop:SubsetCardinality}
	If $X\subset Y$, then $\# X\leq \# Y$.
\end{prop}

\hintbox{
	Directly construct an injection $f:X\to Y$.
	If the codomain of $f$ was $X$, then you could choose a bijection quite easily.
	What happens if you expand the codomain back to $Y$?
}

\subsection*{Countable unions of sets}

We study the cardinality of ordinary and disjoint unions of countable sets.

\begin{lem} \label{lem:CountableUnions1}
	If $\#A=\#B=\aleph_0$, then $\#(A\times B)=\aleph_0$.
\end{lem}

\hintbox{
	Stitch together some bijections that must exist, to produce a useful bijection.
}

\begin{prop} \label{prop:CountableUnions2}
	If, for each $j\in J$, $A_j$ is a nonempty set with cardinality $\#A_j\leq\aleph_0$, and $\#J=\aleph_0$, then
	$$
		\# \bigsqcup_{j\in J} A_j = \aleph_0.
	$$
\end{prop}

\hintbox{
	Start by recalling the definition of disjoint union.
	Find separately an injection and a surjection to the disjoint union of interest, and use the CSB theorem.

	Can you find a surjection from $\NN\times J$ to the disjoint union?

	Note that we asked for all $A_j$ to be nonempty.
	What would happen if all (or all but finitely many) $A_j$ were $\emptyset$?
	This offers a clue to how an injection might be constructed.
}

\begin{prop} \label{prop:CountableUnions3}
	If, for each $j\in J$, $\#A_j\leq\aleph_0$, and $\#J=\aleph_0$, then
	$$
		\# \bigcup_{j\in J} A_j \leq \aleph_0.
	$$
\end{prop}

\hintbox{
	Direct construction of a surjection from the disjoint union of the previous proposition to this ordinary union.
}

\subsection*{Cardinality and power sets}

We say that two sets are \emph{equinumerous} if they have the same cardinality. The following proposition says that the operation ``take the power set'' preserves equinumerosity.

\begin{prop} \label{prop:PowerSetsPreserveEquinumerosity}
	Suppose $\# X=\# Y$. Then $\# \power(X)=\# \power(Y)$. Moreover, if $\# X\leq\# Y$, then $\# \power(X)\leq\# \power(Y)$.
\end{prop}

\hintbox{
	There must be a bijection $f:X\to Y$. Pick an arbitrary $S\subset X$.
	Think about the process of taking the image of $S$ in $f$.
	Is that a function of $S$? Showing injectivity / surjectivity requires you to exploit injectivity / surjectivity of $f$.

	It is probably more efficient to prove the second part first.
}
